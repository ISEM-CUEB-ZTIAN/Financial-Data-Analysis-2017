% Created 2017-04-01 Sat 19:01
% Intended LaTeX compiler: pdflatex
\documentclass[a4paper,11pt]{article}
\usepackage[utf8]{inputenc}
\usepackage[T1]{fontenc}
\usepackage{graphicx}
\usepackage{grffile}
\usepackage{longtable}
\usepackage{wrapfig}
\usepackage{rotating}
\usepackage[normalem]{ulem}
\usepackage{amsmath}
\usepackage{textcomp}
\usepackage{amssymb}
\usepackage{capt-of}
\usepackage{hyperref}
\usepackage[margin=1in]{geometry}
\usepackage{setspace}
\onehalfspacing
\usepackage{parskip}
\usepackage{mathtools}
\usepackage{hyperref}
\hypersetup{colorlinks,citecolor=black,filecolor=black,linkcolor=black,urlcolor=black}
\usepackage{graphicx}
\usepackage{tabularx}
\usepackage{color}
\newtheorem{mydef}{Definition}
\newtheorem{mythm}{Theorem}
\newcommand{\dx}{\mathrm{d}}
\newcommand{\var}{\mathrm{Var}}
\newcommand{\cov}{\mathrm{Cov}}
\newcommand{\corr}{\mathrm{Corr}}
\newcommand{\pr}{\mathrm{Pr}}
\newcommand{\rarrowd}[1]{\xrightarrow{\text{ \textit #1 }}}
\DeclareMathOperator*{\plim}{plim}
\newcommand{\plimn}{\plim_{n \rightarrow \infty}}
\usepackage{marginnote}
\newcommand{\mymarginnote}[2]{\marginnote{\setstretch{1.0}\parbox[t]{\marginparwidth}{\scriptsize \textcolor{#1}{#2}}}}
\usepackage{todo}
\setcounter{secnumdepth}{2}
\author{Zheng Tian}
\date{\today}
\title{Lecture 1. Review on the Linear Time Series Model}
\hypersetup{
 pdfauthor={Zheng Tian},
 pdftitle={Lecture 1. Review on the Linear Time Series Model},
 pdfkeywords={},
 pdfsubject={},
 pdfcreator={Emacs 25.1.1 (Org mode 9.0.3)}, 
 pdflang={English}}
\begin{document}

\maketitle
\setcounter{tocdepth}{1}
\tableofcontents



\section{Introduction}
\label{sec:org0f1543c}

This lecture reviews what we have learned about the linear time series
models, namely, AR, MA, and ARMA models. These models are the
foundation of what we are going to learn in the following lectures. The
review is far from comprehensive but gives you a big picture regarding
these models and links them with what to be learned next. 


\section{Financial Time Series Data}
\label{sec:org325285c}

This course mainly concerns the time series of the returns to
financial assets. Let \(P_t\) be the price of an financial asset at time
\(t\). Then, what we are mostly interested is the following

\[ r_t = ln(P_t) - ln(P_{t-1}) \]

\(\{r_t\}\) is the series of asset return for \(t = 1, \ldots, T\). 


\section{Weak Stationarity}
\label{sec:orgce3229e}

The foundation of time series analysis is the concept of stationarity. Mostly, we
focus on weak stationarity. 

A series \(\{r_t\}\) is weakly stationary if
\begin{enumerate}
\item \(E(r_t) = \mu < \infty\) where \(\mu\) is a constant
\item \(\cov(r_t, r_{t-\ell}) = \gamma_{\ell} < \infty\), which only depends on \(\ell\).
\end{enumerate}

It follows that \(\var(r_t) = \gamma_0 < \infty\), which is also a
constant. 


\section{The ACF and the Ljung-Box Test}
\label{sec:org3906545}

We can use the autocorrelation function (ACF) to characterize the influence
the past value of the series \(r_{t-i}\) for \(i = 1, \ldots, T\) on the
current value \(r_t\). 

\begin{itemize}
\item The lag-\(\ell\) ACF of the series \(\{r_t\}\) is
\[ \rho_{\ell} = \frac{\cov(r_t, r_{t-\ell})}{\sqrt{\var(r_t)\var(r_{t-\ell})}} = \frac{\gamma_{\ell}}{\gamma_0} \]

\item The sample lag-\(\ell\) ACF is computed as 
\[ \hat{\rho}_{\ell} = \frac{\sum_{t=\ell+1}^T (r_t -
  \bar{r})(r_{t-\ell} - \bar{r})}{\sum_{t=1}^T (r_t - \bar{r})^2},\; 0
  \leq \ell < T-1 \]

\item The Ljung-Box test is commonly used to test the existence of
autocorrelation in \(\{r_t\}\). 

The null hypothesis is 
\[ H_0: \rho_1 = \cdots = \rho_m = 0,\; H_1: \rho_i \neq 0 \text{
  for some } i \in \{1, \ldots, m\} \]

The test statistic is 
\[ Q(m) = T(T+2)\sum_{\ell=1}^m \frac{\hat{\rho}^2_{\ell}}{T-\ell}
  \sim \chi^2(m) \]
When \(Q(m) > \chi^2_{\alpha}\), where \(\chi^2_{\alpha}\) is the
critical value at the significance level of \(\alpha\) of a
chi-squared distribution with \(m\) degree of freedom.
\end{itemize}

\subsection{{\bfseries\sffamily TODO} Replicate Figure 2.1 and 2.2}
\label{sec:org798fe60}


\section{The Linear Time Series Models}
\label{sec:orgc696503}

\subsection{The ARMA Model}
\label{sec:orgdcdbf2c}

Autoregressive moving-average models \(\mathrm{ARMA}(p, q)\)
encompass autoregressive models \(\mathrm{AR}(p)\) and
moving-average models \(\mathrm{MA}(q)\). 

A general \(\mathrm{ARMA}(p, q)\) model is in the form of
\begin{equation}
\label{eq:armapq}
r_t = \phi_0 + \sum_{i=1}^p \phi_i r_{t-i} + a_t - \sum_{i=1}^q \theta_i a_{t-i}
\end{equation}
where \(\{a_t\}\) is a white noise series, i.e., \(a_t \sim
\mathrm{i.i.d.}(0, \sigma^2_a)\). 

From the general \(\mathrm{ARMA}(p, q)\) model, we know that \(\mathrm{AR}(p)\) models are
simply \(\mathrm{ARMA}(p, 0)\) and \(\mathrm{MA}(q)\) models are \(\mathrm{ARMA}(0, q)\) for \(p, q >
0\). 

What we are interested in these models can be summarized by the
following items:
\begin{itemize}
\item The stationarity condition
\item The statistical properties
\begin{itemize}
\item The unconditional mean, \(E(r_t)\)
\item The unconditional variance, \(\var(r_t)\).
\item The ACF, \(\rho_{\ell}\) for \(\ell > 0\).
\end{itemize}
\item Estimation and model checking
\item Forecasting
\end{itemize}


\subsection{The stationarity condition}
\label{sec:orga6457de}

\begin{itemize}
\item The characteristic equation of all \(\mathrm{ARMA}(p, q)\) models is
\begin{equation}
\label{eq-chareq}
\alpha^p + \phi_1 \alpha^{p-1} + \cdots + \phi_p = 0
\end{equation}
The solutions to this equation are the characteristic roots.

\item The weak stationarity requires that the characteristic roots be less
than one in modulus. 
\begin{itemize}
\item If the root is a real number, \(\alpha^{*}\), then weak stationarity
requires \(|\alpha^{*}| < 1\).
\item If the root is a complex number, \(\alpha^{*} = a + bi\) where \(i =
    \sqrt{-1}\), then weak stationarity requires \(r = \sqrt{a^2 + b^2} < 1\).
\end{itemize}

\item \(\mathrm{AR}(p)\) and \(\mathrm{ARMA}(p,q)\) share the same
characteristic equation as Equation \eqref{eq-chareq} so that their
stationarity conditions are also the same.

\item \(\mathrm{MA}(q)\) models are always weakly stationary as long as the
\(\{a_t\}\) series is white noise.
\end{itemize}

\subsubsection*{{\bfseries\sffamily TODO} Draw a unit circle}
\label{sec:org1932bf8}


\subsection{The AR Model}
\label{sec:orgc461db7}

We review the properties of \(\mathrm{AR}(p)\) model using the simple
\(\mathrm{AR}(1)\) process, 
\begin{equation}
\label{eq-ar1}
r_t = \phi_0 + \phi_1 r_{t-1} + a_t,\; a_t \sim i.i.d.(0, \sigma^2_a)
\end{equation}

\subsubsection*{The stationarity condition}
\label{sec:org9463fb2}

The characteristic equation of Equation \eqref{eq-ar1} is
\[ \alpha - \phi_1 = 0 \]
The characteristic root is simply \(\alpha = \phi_1\). Thus, the
stationarity condition of an \(\mathrm{AR}(1)\) process is
\(|\phi_1|<1\). 

Remember that when we derive the unconditional mean, variance and ACF
of \(r_t\), we always assume that \(\{r_t\}\) is weakly stationary that is
\(|\phi_1| < 1\). 

\subsubsection*{The expectations}
\label{sec:orgd4a20ad}

\begin{itemize}
\item The unconditional mean of \(r_t\) is
\[ E(r_t) = \mu = \frac{\phi_0}{1 - \phi_1} \]
Because \(\{r_t\}\) is weakly stationary, its mean is constant over
time.

\item The conditional mean of \(r_t\) given the information at \(t-1\) is
\[ E(r_t \mid r_{t-1}) = \phi_0 + \phi_1 r_{t-1} \]
\end{itemize}

\subsubsection*{The variance}
\label{sec:orgdc34cfd}

\begin{itemize}
\item The unconditional variance of \(r_t\) is 
\[ \var(r_t) = \frac{\sigma^2_a}{1 - \phi_1^2} \]
The unconditional variance is also a constant because of weak
stationarity. The existence of the unconditional mean and variance
of \(r_t\) requires \(|\phi_1| < 1\), which is also the sufficient
condition for weak stationarity.

\item The conditional variance of \(r_t\) given \(r_{t-1}\) is
\[ \var(r_t \mid r_{t-1}) = \var(a_t) = \sigma^2_a \]
\end{itemize}

\subsubsection*{The ACF}
\label{sec:org952224a}

The ACF of \(\mathrm{AR}(1)\) is
\[\rho_0 = 1,\; \rho_{\ell} = \phi_1 \rho_{\ell-1}, \text{ for }
\ell>0 \]
It says that the ACF of a weakly stationary AR(1) series decays
exponentially with rate \(\phi_1\) and starting value \(\rho_0=1\). 

\begin{itemize}
\item {\bfseries\sffamily TODO} Insert Figure 2.3
\label{sec:org8f1e13d}
\end{itemize}

\subsubsection*{The ACF for the general AR(p) model}
\label{sec:orgd956dff}

For a general \(AR(p)\) model, 
\begin{equation}
\label{eq-arp}
r_t = \phi_0 + \sum_{i=1}^p \phi_i r_{t-i} + a_t,\; a_t \sim i.i.d.(0, \sigma^2_a)
\end{equation}

\begin{itemize}
\item The unconditional mean is 
\[ E(r_t) = \frac{\phi_0}{1 - \sum_{i=1}^p \phi_i} \]

\item The ACF of \{\(r_t\)\} is governed by the following difference equation
\[ \rho_{\ell} = \phi_1 \rho_{\ell-1} + \phi_2 \rho_{\ell-2} +
  \cdots + \rho_{\ell-p} \]
Rewritten with the lag operator \(B\), we have
\[ (1 - \phi_1 B - \cdots - \phi_p B^p) \rho_{\ell} = 0 \]
where \(1 - \phi_1 B - \cdots - \phi_p B^p=0\) is the inverse
characteristic equation.

\item An \(\mathrm{AR}(p)\) series is weakly stationary when all the roots
of the inverse characteristic equation are greater than one in
modulus.
\end{itemize}


\subsection{{\bfseries\sffamily TODO} The MA Model}
\label{sec:org8f7c15f}



\section{{\bfseries\sffamily TODO} Random Walk and Unit-Root Nonstationarity}
\label{sec:orgb7ed07b}



\section{{\bfseries\sffamily TODO} The Basic R functions for Financial Data}
\label{sec:orgb75e331}
\end{document}