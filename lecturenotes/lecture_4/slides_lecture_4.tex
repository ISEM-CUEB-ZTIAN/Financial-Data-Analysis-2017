% Created 2017-05-27 Sat 13:09
% Intended LaTeX compiler: pdflatex
\documentclass[presentation,10pt]{beamer}
\usepackage[utf8]{inputenc}
\usepackage[T1]{fontenc}
\usepackage{graphicx}
\usepackage{grffile}
\usepackage{longtable}
\usepackage{wrapfig}
\usepackage{rotating}
\usepackage[normalem]{ulem}
\usepackage{amsmath}
\usepackage{textcomp}
\usepackage{amssymb}
\usepackage{capt-of}
\usepackage{hyperref}
\usepackage{amsthm}
\usepackage{amsmath}
\usepackage{amssymb}
\usepackage{mathtools}
\newtheorem{mydef}{Definition}
\newtheorem{mythm}{Theorem}
\newcommand{\dx}{\mathrm{d}}
\newcommand{\var}{\mathrm{Var}}
\newcommand{\cov}{\mathrm{Cov}}
\newcommand{\corr}{\mathrm{corr}}
\newcommand{\pr}{\mathrm{Pr}}
\newcommand{\rarrowd}[1]{\xrightarrow{\text{ \textit #1 }}}
\DeclareMathOperator*{\plim}{plim}
\newcommand{\plimn}{\plim_{n \rightarrow \infty}}
\usepackage{booktabs}
\usepackage{color}
\usepackage{caption}
\usepackage{subcaption}
\def\mathbi#1{\textbf{\em #1}}
\setlength{\parskip}{1em}
\newcommand{\undersetdisp}[2]{\underset{\displaystyle #1}{#2}}
\usetheme{CambridgeUS}
\usecolortheme{beaver}
\author{Zheng Tian}
\date{}
\title{Lecture 4: Alternative Volatility Models}
\hypersetup{
 pdfauthor={Zheng Tian},
 pdftitle={Lecture 4: Alternative Volatility Models},
 pdfkeywords={},
 pdfsubject={},
 pdfcreator={Emacs 25.1.1 (Org mode 9.0.3)}, 
 pdflang={English}}
\begin{document}

\maketitle
\begin{frame}{Outline}
\tableofcontents
\end{frame}



\section{The IGARCH Model}
\label{sec:orgbbb9d90}

\section{The GARCH-M Model}
\label{sec:org1fa1522}

\section{Models with Asymmetry: EGARCH and TGARCH}
\label{sec:orgb56056f}

\begin{frame}[label={sec:orgfee991a}]{EGARCH(m, s): general case}
\begin{itemize}
\item EGARCH(m, s)

\begin{align*}
a_t &= \sigma_t \epsilon_t,\; \epsilon_t \sim i.i.d.(0, 1) \\
\ln(\sigma^2_t)&= \alpha_0 + \frac{1 + \beta_1 B + \cdots + \beta_{s-1} B^{s-1}}{1 - \alpha_1 B - \cdots - \alpha_m B^m} g(\epsilon_{t-1})
\end{align*}

\item Asymmetry via \(g(\epsilon_t)\)

\begin{align*}
g(\epsilon_t) &= \theta \epsilon_t + \gamma \left[ \left| \epsilon_t \right| - E \left( |\epsilon_t| \right)  \right] \\
&=
\begin{cases}
(\theta + \gamma)\epsilon_t - \gamma E(|\epsilon_t|), \text{ when } \epsilon_t \geq 0  \\
(\theta - \gamma)\epsilon_t - \gamma E(|\epsilon_t|), \text{ when } \epsilon_t < 0
\end{cases}
\end{align*}
\end{itemize}
\end{frame}

\begin{frame}[shrink,label={sec:org599b799}]{Example: EGARCH(1, 1)}
\begin{itemize}
\item EGARCH(1, 1)
\begin{align*}
a_t &= \sigma_t \epsilon_t,\; \epsilon_t \sim i.i.d. N(0, 1) \text{ and } E(|\epsilon_t|) = \sqrt{2/\pi} \\ \\
\ln(\sigma^2_t)&= \alpha_0 + \frac{1}{1 - \alpha B} g(\epsilon_{t-1}) \\
g(\epsilon_t) &= \theta \epsilon_t + \gamma \left( |\epsilon_t| - \sqrt{2/\pi} \right)
\end{align*}

\item Derive \(\sigma^2_t\)
\begin{equation*}
(1 - \alpha B) \ln(\sigma^2_t) = 
\begin{cases}
\alpha^{*} + (\theta + \gamma) \epsilon_{t-1},\; \text{ when } \epsilon_{t-1} \geq 0 \\
\alpha^{*} + (\theta - \gamma) \epsilon_{t-1},\; \text{ when } \epsilon_{t-1} < 0
\end{cases}
\end{equation*}
where \(\alpha^{*} = \alpha_0(1 - \alpha) - \gamma
  \sqrt{2/\pi}\). Therefore, 
\begin{equation*}
\sigma^2_t = \sigma^2_{t-1} \exp(\alpha^{*}) 
\begin{cases}
\exp \left( (\gamma+\theta) \frac{a_{t-1}}{\sigma_{t-1}} \right) \\
\exp \left( (\gamma-\theta) \frac{|a_{t-1}|}{\sigma_{t-1}} \right)
\end{cases}
\end{equation*}
\end{itemize}
\end{frame}

\section{The CHARMA model}
\label{sec:org5e505bb}

\begin{frame}[label={sec:orga1f3b54}]{What is a CHARMA model?}
\begin{itemize}
\item CHARMA: Conditional heteroskedasticity ARMA model (Tsay, 1987)
\end{itemize}

\vspace{0.05cm}

\begin{itemize}
\item A general CHARMA model
\begin{align*}
r_t &= \mu_t + a_t \\
a_t &= \delta_{1t} a_{t-1} + \delta_{25} a_{t-2} + 
       \cdots + \delta_{mt} a_{t-m} + \eta_t
\end{align*}

\item Gaussian white noise: \(\eta_t \sim i.i.d. N(0, \sigma^2_{\eta})\)
\end{itemize}

\vspace{0.05cm}

\begin{itemize}
\item Random coefficient: \(\boldsymbol{\delta}_t = (\delta_{1t},
  \delta_{2t}, \ldots, \delta_{mt})^{\prime}\).
\end{itemize}

\vspace{0.05cm}

\begin{itemize}
\item Let \(\mathrm{a}_{t-1} = (a_{t-1}, \ldots, a_{t-m})\). Then CHARMA
model is rewritten as 
\[ a_t = \mathrm{a}_{t-1}^{\prime} \boldsymbol{\delta}_t + \eta_t \]
\end{itemize}
\end{frame}

\begin{frame}[label={sec:orgbb8d9d2}]{Random coefficients}
\begin{itemize}
\item \(\{\boldsymbol{\delta}_t\} = \{(\delta_{1t}, \delta_{2t}, \ldots,
  \delta_{mt})^{\prime}\}\) is a sequence of i.i.d. random vectors.
\end{itemize}

\vspace{0.1cm}

\begin{itemize}
\item \(\{\boldsymbol{\delta}_t\}\) is independent of \(\{\eta_t\}\).
\end{itemize}

\vspace{0.1cm}

\begin{itemize}
\item \(E(\boldsymbol{\delta}_t) = \mathbf{0}\) and \(E(\boldsymbol{\delta}_t
  \boldsymbol{\delta}_t^{\prime}) = \boldsymbol{\Omega}\).
\begin{equation*}
\boldsymbol{\Omega} = 
\begin{pmatrix}
\omega_{11} & \omega_{12} & \cdots & \omega_{1m} \\
\omega_{21} & \omega_{22} & \cdots & \omega_{2m} \\
\vdots      & \vdots      & \ddots & \vdots \\
\omega_{m1} & \omega_{m2} & \cdots & \omega_{mm} \\
  \end{pmatrix}
\end{equation*}
\end{itemize}
\end{frame}

\begin{frame}[label={sec:org7e1926a}]{Conditional variance of \(a_t\) in CHARMA}
\begin{itemize}
\item The conditional variance of \(a_t\), i.e. \(\var_{t-1}(a_t) = \sigma^2_t\):
\begin{equation*}
\begin{split}
  \sigma^2_t &= \sigma^2_{\eta} + \mathbf{a}_{t-1}^{\prime} \boldsymbol{\Omega} \mathbf{a}_{t-1} \\
             &= \sigma^2_{\eta} + (a_{t-1}, \ldots, a_{t-m})\boldsymbol{\Omega} (a_{t-1}, \ldots, a_{t-m})^{\prime}
\end{split}
\end{equation*}
\end{itemize}

\vspace{0.1cm}

\begin{itemize}
\item Since \(\boldsymbol{\Omega}\) is positive semi-definite, \(\sigma^2_t
  \geq \sigma^2_{\eta} > 0\) is always true.
\end{itemize}

\vspace{0.1cm}

\begin{itemize}
\item The conditional variance is similar to ARCH but with difference.

\begin{itemize}
\item When \(m = 1\), \(\sigma^2_t = \sigma^2_{\eta} + \omega_{11} a_{t-1}^2\).

\item When \(m =2\), \(\sigma^2_t = \sigma^2_{\eta} + \omega_{11} a_{t-1}^2 +
    \underbrace{2\omega_{12} a_{t-1} a_{t-2}}_{\text{cross-product term}} + \omega_{22} a^2_{t-2}\).
\end{itemize}
\end{itemize}
\end{frame}

\begin{frame}[label={sec:org09ff7b6}]{Problem of CHARMA}
\begin{itemize}
\item The number of cross-product terms increases with the order of \(m\).
\end{itemize}

\vspace{0.1cm}

\begin{itemize}
\item The higher-order properties are hard to derive.
\end{itemize}
\end{frame}

\section{Random Coefficient Autoregressive Model (RCA)}
\label{sec:org98ba0cb}

\begin{frame}[label={sec:org2da6093}]{Why use random coefficient autoregressive model?}
\begin{itemize}
\item Random coefficient: account for variability among different subjects.
\begin{itemize}
\item Panel data; hierarchical models
\end{itemize}
\end{itemize}

\vspace{0.2cm}

\begin{itemize}
\item Time series model: the coefficients in the mean equation evolve over time.
\end{itemize}
\end{frame}

\begin{frame}[label={sec:org445e853}]{RCA(p)}
\begin{itemize}
\item The mean equation

\begin{equation*}
r_t = \phi_0 + \sum_{i=1}^p (\phi_i + \delta_{it}) r_{t-i} + a_t
\end{equation*}

\item Random coefficients

\[\{\boldsymbol{\delta}_t\} = \{(\delta_{1t}, \delta_{2t}, \ldots,
  \delta_{pt})^{\prime}\}\]

\begin{itemize}
\item Independent series;
\item \(E(\boldsymbol{\delta}_t) = \mathbf{0}\) and
\(\var{\boldsymbol{\delta}_t} = \boldsymbol{\Omega}_{\delta}\);
\item \(\{\boldsymbol{\delta}_t\}\) is independent of \(\{\mathbf{a}_t\}\)
\end{itemize}
\end{itemize}
\end{frame}

\begin{frame}[label={sec:orga3c595c}]{Conditional mean and variance}
\begin{itemize}
\item The conditional mean

\[\mu_t = E_{t-1}(r_t) = \phi_0 + \sum_{i=1}^p \phi_i r_{t-i} \]

\item The conditional variance

\begin{equation*}
\begin{split}
\sigma^2_t &= E_{t-1} \left((r_t - \mu_t)^2 \right) = E_{t-1} \left((\mathbf{r}_{t-1})^{\prime} \boldsymbol{\delta}_t + a_t)^2 \right) \\
&= \sigma^2_a + (r_{t-1}, \ldots, r_{t-p}) \boldsymbol{\Omega}_{\delta} (r_{t-1}, \ldots, r_{t-p})^{\prime}
\end{split}
\end{equation*}

\item Similar to CHARMA but with the quadratic function of \(r_{t-i}\).
\end{itemize}
\end{frame}
\end{document}